%%%%%%%%%%%%%%%%%%%%%%%%%%%%%%%%%%%%%%%%%%%%%%%%%%%%%%%%%%%%%%%%%%%%%%%%%%%%%%%%
\section{Related Work}

\subsection{High-Performance Computation}

\par TODO\citep{compiler-for-hpc}

\subsection{Community Detection Algorithms}

\par TODO

% What is AGM? How it works? Its Experiement results?
% This is a convex optimization 😃 :)
\paragraph{AGM} \textit{Community-Affiliation Graph Model}\citep{yang2012structure}
is a conceptual model of network community structure, which detects 
\textbf{overlapping} network communities. Given a network graph $G(V,E)$, the
goal is to create a bipartite affiliation graph $B(V, C, M)$, where
$u\in V$ is a node(vertex), $c\in C$ is a community, 
$k\in M$ is an edge $(u,c)$ if $u$ belongs to community $c$. Each
community $c\in C$ has a probability $p_c\in [0,1]$.
For each edges $(u,v)\in G(E), u,v \in V$, the likelihood of it is 
\begin{align}
    p_(u,v) &= 1- \prod_{k\in C_{uv}}(1-p_k)
\end{align},
where $C_{uv}\subset C$ is a set of communities $u,v$ have in common.
The community detection problem is now to optimize the likelihood of 
observed edges $G(E)$: 
\begin{align} \label{form:agmopt}
    \text{argmax}_{\{p_c\}} L(\{p_c\}) &= \prod_{(u,v)\in E} p(u,v) 
        \prod_{(u,v)\notin E} (1 - p(u,v))\\
    &\equiv \sum_{(u,v)\in E}\log{(1 - \prod_{k\in C_{uv}} (1-p_k))} + 
        \sum_{(u,v)\notin E}\log{\prod_{k\in C_{uv}} (1-p_k)}
\end{align}
Problem~\ref{form:agmopt}\footnote{It can be seen that $\{p_c\}$ is the parameters to fit.}
can be converted to a convex optimization 
problem and solved efficiently\citep{yang2012structure}. 
With this model, Yang et al.\citep{yang2012structure} concluded that overlaps of communities are
usually densely connected, and that ground-truth communities contain
high-degree hub nodes that reside in community overlaps.

\subsection{Node Centrality Algorithms}
\par TODO
