%%%%%%%%%%%%%%%%%%%%%%%%%%%%%%%%%%%%%%%%%%%%%%%%%%%%%%%%%%%%%%%%%%%%%%%%%%%%%%%%
\section{Related Work}

\subsection{High-Performance Computation}

\paragraph{Multi-Processor Programming Frameworks} POSIX-Thread\citep{pthread}
(or Pthreads) and OpenMP\citep{openmp} are two commonly-used shared-memory 
multi-processor parallel programming frameworks.  A single process can have 
several POSIX threads, executing certain parts of the program in 
parallel. The programmers have to explicitly spawn or destroy POSIX threads.
OpenMP, on the other hand, implicitly manages threads, therefore
offering programmers a simple \texttt{\#pragma omp} directive to parallel execution of 
loops\footnote{An example of this is in Figure~\ref{code:ompdemo}}.
Our thread pool implementation discussed in Section~\ref{sec:impl:tp}
is based on POSIX-Thread because it offers explicit way to 
synchronize the worker and host threads.

\paragraph{Compiler Support for HPC} Compiler optimizations  has become an
integral part of modern high-performance computer systems. As these 
optimizations become increasingly effective and versatile, programmers
can be less concerned about the underlying intricate details of hardware
architectures. Most optimizations for uni-processors reduce the number of
instructions executed by the program using program analysis based 
transformations\citep{compiler-for-hpc}. However, since there exist several
different parallel computing architectures(vector-CPU, shared-memory
multi-processor, distributed-memory multi-processor), automatically 
parallelize code remains highly challenging for compilers.

% "😭 -> 😃" 

\subsection{Community Detection Algorithms}

\par We investigate two types of community detection algorithms,
Community Structure\citep{girvan2002community}  and AGM\citep{yang2012structure}.

\paragraph{Community Structured Based Method} A conspicuous property of 
community structure is that network nodes are tightly joined, and that
there are only loose connections between communities. Based on this,
Girvan et al. proposed a method to detect such community structure and
find non-overlapping communities in both social and biological networks
\citep{girvan2002community}. It iteratively computes the "betweenness"\citep{freeman1977graph}
of each edge, removes the edge with highest "betweenness", and terminates
when there is no edges left. This method captures community structure
with high accuracy and sensitivity. However, betweenness calculation algorithm's 
complexity  is $O(|E| |V|)$ for network $G(V,E)$, resulting in an
entire worst-case time of $O(|E|^2 |V|)$.

% Amdahl's law :)

% What is AGM? How it works? Its Experiement results?
% This is a convex optimization 😃 :)
\paragraph{AGM} \textit{Community-Affiliation Graph Model}\citep{yang2012structure}
is a conceptual model of network community structure, which detects 
\textbf{overlapping} network communities. Given a network graph $G(V,E)$, the
goal is to create a bipartite affiliation graph $B(V, C, M)$, where
$u\in V$ is a node(vertex), $c\in C$ is a community, 
$k\in M$ is an edge $(u,c)$ if $u$ belongs to community $c$. Each
community $c\in C$ has a probability $p_c\in [0,1]$.
For each edges $(u,v)\in G(E), u,v \in V$, the likelihood of it is 
\begin{align}
    p_(u,v) &= 1- \prod_{k\in C_{uv}}(1-p_k)
\end{align},
where $C_{uv}\subset C$ is a set of communities $u,v$ have in common.
The community detection problem is now to optimize the likelihood of 
observed edges $G(E)$: 
\begin{align} \label{form:agmopt}
    \text{argmax}_{\{p_c\}} L(\{p_c\}) &= \prod_{(u,v)\in E} p(u,v) 
        \prod_{(u,v)\notin E} (1 - p(u,v))\\
    &\equiv \sum_{(u,v)\in E}\log{(1 - \prod_{k\in C_{uv}} (1-p_k))} + 
        \sum_{(u,v)\notin E}\sum_{k\in C_{uv}} \log{(1-p_k)}
\end{align}
Problem~\ref{form:agmopt}\footnote{It can be seen that $\{p_c\}$ is the parameters to fit.}
can be converted to a convex optimization 
problem and solved efficiently\citep{yang2012structure}. 
With this model, Yang et al.\citep{yang2012structure} concluded that overlaps of communities are
usually densely connected, and that ground-truth communities contain
high-degree hub nodes that reside in community overlaps.

\subsection{Node Centrality Algorithms}
\par TODO
