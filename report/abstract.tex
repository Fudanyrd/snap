\begin{abstract}
  Stanford Network Analysis Platform (\textbf{SNAP}) is a general purpose, high
  performance system for analysis and manipulation of large networks.
  \textbf{SNAP} is written in \texttt{C++} and it scales to massive graphs with hundreds
  of millions of nodes and billions of edges. However, when applying the \textbf{SNAP}
  framework to real-world datasets, we found that the performance of \textbf{SNAP}
  is highly constrained by its single-thread execution. To this end, we have made
  manual efforts to optimize its scalability and ultimately, its runtime overhead 
  on various social-network mining tasks. Our methodology includes crating a high-performance 
  thread pool and utilize it to parallelize computation-intense code. Experimental 
  results showed that our implementation is \texttt{0.87x} faster on community discovery tasks,
  and \texttt{FIXME-x} faster for node centrality calculation.
\end{abstract}

