\documentclass[acmlarge]{acmart}
%%
%% \BibTeX command to typeset BibTeX logo in the docs
\AtBeginDocument{%
  \providecommand\BibTeX{{%
    Bib\TeX}}}

%% Rights management information.  This information is sent to you
%% when you complete the rights form.  These commands have SAMPLE
%% values in them; it is your responsibility as an author to replace
%% the commands and values with those provided to you when you
%% complete the rights form.
\setcopyright{acmlicensed}
\copyrightyear{2018}
\acmYear{2018}
\acmDOI{NHwF6SCV.H0d63yPS}

%%
%% These commands are for a JOURNAL article.
\acmJournal{POMACS}
\acmVolume{37}
\acmNumber{4}
\acmArticle{111}
\acmMonth{8}

%%
%% Submission ID.
%% Use this when submitting an article to a sponsored event. You'll
%% receive a unique submission ID from the organizers
%% of the event, and this ID should be used as the parameter to this command.
%%\acmSubmissionID{123-A56-BU3}

%%
%% For managing citations, it is recommended to use bibliography
%% files in BibTeX format.
%%
%% You can then either use BibTeX with the ACM-Reference-Format style,
%% or BibLaTeX with the acmnumeric or acmauthoryear sytles, that include
%% support for advanced citation of software artefact from the
%% biblatex-software package, also separately available on CTAN.
%%
%% Look at the sample-*-biblatex.tex files for templates showcasing
%% the biblatex styles.
%%

%%
%% The majority of ACM publications use numbered citations and
%% references.  The command \citestyle{authoryear} switches to the
%% "author year" style.
%%
%% If you are preparing content for an event
%% sponsored by ACM SIGGRAPH, you must use the "author year" style of
%% citations and references.
%% Uncommenting
%% the next command will enable that style.
%%\citestyle{acmauthoryear}

\usepackage{amsmath}
\usepackage{amssymb}
\usepackage{graphicx}
\usepackage{listings}
\usepackage{multirow}
\usepackage{natbib}
\usepackage{setspace}
\usepackage{subcaption}
\usepackage{url}

%% package listings
\lstset{
  language=C,
  basicstyle=\footnotesize\ttfamily,
  captionpos=b,
  breaklines=false,
  escapeinside={(*}{*)},
  keywordstyle=\color{blue},
  tabsize=2,
  showspaces=false,
  breakatwhitespace=true,
  showstringspaces=false,
  columns=fullflexible,
  numbers=left,
  numbersep=3pt,
  showtabs=false,
  % numberstyle=\scriptsize\ttfamily\color{mygray},
  morekeywords={string,foreach,assert,async,await,lock,pragma,asm,volatile,class}
}
\newcommand{\id}[1]{\texttt{#1}}
\newcommand{\namedplaceholder}[1]{\colorbox{blue}{\textcolor{white}{\id{~#1~}}}}
\newcommand{\placeholder}[1]{\colorbox{yellow}{\textcolor{black}{\id{#1}}}}
\newcommand{\snap}{\textbf{SNAP}\citep{snap}}

%%
%% end of the preamble, start of the body of the document source.
\begin{document}

%%
%% The "title" command has an optional parameter,
%% allowing the author to define a "short title" to be used in page headers.
\title{Faster SNAP: A High-Performance Social Network Mining Framework}

%%
%% The "author" command and its associated commands are used to define
%% the authors and their affiliations.
%% Of note is the shared affiliation of the first two authors, and the
%% "authornote" and "authornotemark" commands
%% used to denote shared contribution to the research.
\author{Rundong Yang}
\authornote{Both authors contributed equally to this research.}
\email{22307140045@m.fudan.edu.cn}
\author{Xitong Wang}
\authornotemark[1]
\email{22307140017@m.fudan.edu.cn}
\affiliation{%
  \institution{Fudan University}
  \city{Shanghai}
  \state{Shanghai}
  \country{China}
}

\citestyle{acmauthoryear}

\begin{abstract}
  Stanford Network Analysis Platform (\textbf{SNAP}) is a general purpose, high
  performance system for analysis and manipulation of large networks.
  \textbf{SNAP} is written in \texttt{C++} and it scales to massive graphs with hundreds
  of millions of nodes and billions of edges. However, when applying the \textbf{SNAP}
  framework to real-world datasets, we found that the performance of \textbf{SNAP}
  is highly constrained by its single-threaded execution. To this end, we have made
  manual efforts to optimize its scalability and ultimately, its runtime overhead 
  on various social network mining tasks. Our methodology includes crating a high-performance 
  thread pool and utilize it to parallelize computation-intense code. Experimental 
  results showed that our implementation is \texttt{1.87x} faster on community detection tasks,
  and \texttt{FIXME-x} faster for node centrality calculation.
\end{abstract}



\begin{CCSXML}
<ccs2012>
   <concept>
       <concept_id>10010147.10011777.10011778</concept_id>
       <concept_desc>Computing methodologies~Concurrent algorithms</concept_desc>
       <concept_significance>500</concept_significance>
       </concept>
 </ccs2012>
\end{CCSXML}

\ccsdesc[500]{Computing methodologies~Concurrent algorithms}

%%
%% Keywords. The author(s) should pick words that accurately describe
%% the work being presented. Separate the keywords with commas.
\keywords{High-Performance Computation, Social Network Mining, Concurrent Algorithms}

\received{09 January 2026}
\received[revised]{09 January 2026}
\received[accepted]{12 March 2027}

% Code Repository

\maketitle
% \setstretch{1.5}

\section{Introduction}
\par Our code is publicly available at \url{https://github.com/Fudanyrd/snap}.

%%%%%%%%%%%%%%%%%%%%%%%%%%%%%%%%%%%%%%%%%%%%%%%%%%%%%%%%%%%%%%%%%%%%%%%%%%%%%%%%
\section{Related Work}

\subsection{High-Performance Computation}

\par TODO\citep{compiler-for-hpc}

\subsection{Community Detection Algorithms}

\par We investigate two types of community detection algorithms,
Community Structure\citep{girvan2002community}  and AGM\citep{yang2012structure}.

\paragraph{Community Structured Based Method} A conspicuous property of 
community structure is that network nodes are tightly joined, and that
there are only loose connections between communities. Based on this,
Girvan et al. proposed a method to detect such community structure and
find non-overlapping communities in both social and biological networks
\citep{girvan2002community}. It iteratively computes the "betweeness"\citep{freeman1977graph}
of each edge, removes the edge with highest "betweeness", and terminates
when there is no edges left. This method captures community structure
with high accuracy and sensitivity. However, betweeness calcluation algorithm's 
complexity  is $O(|E| |V|)$ for network $G(V,E)$, resulting in an
entire worst-case time of $O(|E|^2 |V|)$.

% Amdahl's law :)

% What is AGM? How it works? Its Experiement results?
% This is a convex optimization 😃 :)
\paragraph{AGM} \textit{Community-Affiliation Graph Model}\citep{yang2012structure}
is a conceptual model of network community structure, which detects 
\textbf{overlapping} network communities. Given a network graph $G(V,E)$, the
goal is to create a bipartite affiliation graph $B(V, C, M)$, where
$u\in V$ is a node(vertex), $c\in C$ is a community, 
$k\in M$ is an edge $(u,c)$ if $u$ belongs to community $c$. Each
community $c\in C$ has a probability $p_c\in [0,1]$.
For each edges $(u,v)\in G(E), u,v \in V$, the likelihood of it is 
\begin{align}
    p_(u,v) &= 1- \prod_{k\in C_{uv}}(1-p_k)
\end{align},
where $C_{uv}\subset C$ is a set of communities $u,v$ have in common.
The community detection problem is now to optimize the likelihood of 
observed edges $G(E)$: 
\begin{align} \label{form:agmopt}
    \text{argmax}_{\{p_c\}} L(\{p_c\}) &= \prod_{(u,v)\in E} p(u,v) 
        \prod_{(u,v)\notin E} (1 - p(u,v))\\
    &\equiv \sum_{(u,v)\in E}\log{(1 - \prod_{k\in C_{uv}} (1-p_k))} + 
        \sum_{(u,v)\notin E}\sum_{k\in C_{uv}} \log{(1-p_k)}
\end{align}
Problem~\ref{form:agmopt}\footnote{It can be seen that $\{p_c\}$ is the parameters to fit.}
can be converted to a convex optimization 
problem and solved efficiently\citep{yang2012structure}. 
With this model, Yang et al.\citep{yang2012structure} concluded that overlaps of communities are
usually densely connected, and that ground-truth communities contain
high-degree hub nodes that reside in community overlaps.

\subsection{Node Centrality Algorithms}
\par TODO

%%%%%%%%%%%%%%%%%%%%%%%%%%%%%%%%%%%%%%%%%%%%%%%%%%%%%%%%%%%%%%%%%%%%%%%%%%%%%%%%
\section{Implementation} \label{sec:impl}

\par We present our methodology of discovering the bottle-necks of 
\texttt{Snap v6.0} implementation, and our approach to alleviate the performance
impact via parallelism. Furthermore, to avoid having to constantly spawn and
destroy threads, we designed a generic thread pool to coordinate and 
synchronize these threads to avoid the overhead introduced by thread creation
and deallocation.


\subsection{Performance Analysis of \textbf{SNAP}}

% Snap (version 6.0) is a large code base with about \texttt{111k} 
% lines of \texttt{C++} code.

\par We use \texttt{perf}\citep{linux-perf}, a performance analysis tools to 
guide our rewrite process. It utilize \textbf{performance counters} for Linux,
which are a kernel-based subsystem that provide a framework for all things 
performance analysis. This subsystem sends regular timer interrupts to the
running process, and backtrace of the process's call-stack for the instruction
being executed in its interrupt handler. Such profiling trace provides us with 
runtime statistics of which instructions and functions consume more processor 
time, and ultimately insights
into which part of code to optimize for highest performance improvement.

\paragraph{An overview of snap runtime statistics}. 

\subsection{Thread Pool Implementation}

\par Indeed, the \snap library has some parallelized code, mainly 
with the \texttt{openmp} framework\citep{openmp,embedded-hps}. An example
of such optimization is shown in Figure~\ref{code:openmp}. As part of the page-rank
implementation, this snippet copies results from \texttt{PRankV} to \texttt{PRankH}.
Instead of using one core to move the data, it uses \texttt{openmp} parallelism; 
each  thread copies a disjoint part of data
\footnote{Permalink 
\url{https://github.com/Fudanyrd/snap/blob/b62ece5fe169686a81dbc1a58669722d81114b33/snap-core/centr.cpp\#L518}}.

\begin{figure}[ht]
    \centering
\begin{lstlisting}[frame=tlbr]
    /* Code snippet, line 515, snap-core/centr.cpp */
    #pragma (*\namedplaceholder{omp parallel}*) for schedule(dynamic,100000)
    for (int i = 0; i < NNodes; i++) {
        TNEANet::TNodeI NI = NV[i];
        PRankH[i] = PRankV[NI.GetId()];
    }

    return 0;
\end{lstlisting}
    \caption{An example of \texttt{openmp}-introduced parallelism.}
    \label{code:openmp}
\end{figure}

\par Unfortunately, if the above code snippet is executed intensely, the 
performance improvements introduced by \texttt{openmp} will diminish. Since 
the profiler cannot provide us with the reason for this phenomenon, we can
only assume from empirical evidence that it is because of the thread creation 
and deallocation overhead. An illustrative code snippet and its runtime
statistics is shown in Figure~\ref{code:ompdemo} and Figure~\ref{fig:ompdemo} respectively,
to support this claim\footnote{Compiled and linked with \texttt{-O1 -fopenmp -lrt}}.
Therefore, we feel that keeping all these thread in a global pool is necessary;
it avoids such runtime overhead by creating threads at program initialization,
and destroys them before exiting.

\begin{figure}[ht]
    \centering
\begin{lstlisting}[frame=tlbr]
#include <omp.h>

int main(int argc, char **argv) {
  int fd = open("/dev/null", O_WRONLY);
  /* Here N is provided by command-line arguments. 
   * It controls how many times the sum operation
   * are executed.
   */
  for (int run = 0; run < (*\namedplaceholder{N}*) ; run++) {
    int total = 0;
    #pragma (*\namedplaceholder{omp parallel}*) for
    for (int i = 0; i < 1024; i++) {
      total += rand();
    }
    /* Avoid compiler optimizing out the total,
     * and introduce lower overhead.
     */
    (void) write(fd, &total, sizeof(total));
  }

  close(fd);
  return 0;
}
\end{lstlisting}
    \caption{Demo program to illustrate overhead of threading.}
    \label{code:ompdemo}
\end{figure}

\begin{figure}[ht]
    \centering
    \includegraphics[width=0.5\textwidth]{images/ompdemo.pdf}
    \caption{The Execution Time of Program~\ref{code:ompdemo}. \\
        The \texttt{Ideal} line illustrates the execution time when the threading
        approach has no overhead. \\
        We conclude that the overhead of threading increases with scaling of $N$.
        When $N = 10^5$, the overhead reaches \textbf{58.2\%}. }
    \label{fig:ompdemo}
\end{figure}

\par The idea of the thread pool is to reserve several worker threads; when the
main thread needs to perform some intense computations, the pool will wake up
these workers to perform their assigned tasks separately. The pseudo code of how
we want to parallelize the snippet~\ref{code:ompdemo} with a thread pool
is shown in Figure~\ref{code:tpdemo}. Task assignment is done by the \texttt{addTask}
method; retrieving result is achieved by \texttt{waitFor}.

\begin{figure}[ht]
    \centering
\begin{lstlisting}[frame=tlbr]
/* Global thread pool. */
extern (*\namedplaceholder{ThreadPool}*) tpool;

/* Returns sum of `n` random numbers. */
static int worker(int n) {
    int ret = 0;
    for (int i = 0; i < n; i++) ret += rand();
    return ret;
}

int main() {
    /* Divide the task evenly into K parts; 
     * add tasks the pool for worker thread to perform. 
     */
    for (int t = 0; t < K; t++) (*\namedplaceholder{tpool.addTask}*)(worker, N / K);

    int total = 0, claimed = 0;
    task_t *finished;
    /* Aggregate the result of each task. */
    while (finished = (*\namedplaceholder{tpool.waitFor}*)()) 
        (total += TASK_RESULT(finished), claimed += 1);
    
    /* Should claim K results. */
    assert(claimed == K);
}
\end{lstlisting}
\end{figure}

%%%%%%%%%%%%%%%%%%%%%%%%%%%%%%%%%%%%%%%%%%%%%%%%%%%%%%%%%%%%%%%%%%%%%%%%%%%%%%%%


%%%%%%%%%%%%%%%%%%%%%%%%%%%%%%%%%%%%%%%%%%%%%%%%%%%%%%%%%%%%%%%%%%%%%%%%%%%%%%%%
\section{Evaluation}\label{sec:evaluation}

\paragraph{Experiment Setup}. We ran our experiments on a physical
\texttt{Ubuntu-22.04} server with 8 Intel i5 cores and 16 GB RAM. We use 
\texttt{x86\_64-linux-gnu-g++} version \texttt{11.4.0} as our \texttt{C++} compiler.
We build the \textbf{SNAP} project in debug mode
\footnote{This means that assertions and debug outputs in the code are enabled.}
and with highest possible compiler optimization  level(\texttt{-O3}), and 
enabled link-time optimization(\texttt{-flto}).
For parallel executions, we use $6$ worker threads when creating the thread
pool
\footnote{Along with the main(coordinator) thread, a total of 7 threads 
are used.}.

\paragraph{Evaluation Metrics}. We expect our implementation to achieve 
performance gains via parallelized execution. Therefore, we compare the 
\textbf{execution time} and \textbf{CPU utility rate} of our 
manually-optimized code. Ideally, a thread pool with $N$ workers should yield
a CPU utility rate of $N\times 100\%$, and a speed-up of $N$\texttt{x}. However,
we did not try to evaluate the \texttt{scalability}, a desirable property of 
parallel algorithms, because of the limitation of computation capabilities of
 our physical server.

\subsection{Community-Affiliation Graph Model}

\par Community-Affiliation Graph Model\citep{yang2012structure,yang2012community} 
is a conceptual model of community network, capturing the overlapping nature of 
network communities.

%%%%%%%%%%%%%%%%%%%%%%%%%%%%%%%%%%%%%%%%%%%%%%%%%%%%%%%%%%%%%%%%%%%%%%%%%%%%%%%%



\bibliographystyle{ACM-Reference-Format}
\bibliography{reference}

\end{document}
