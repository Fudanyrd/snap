%%%%%%%%%%%%%%%%%%%%%%%%%%%%%%%%%%%%%%%%%%%%%%%%%%%%%%%%%%%%%%%%%%%%%%%%%%%%%%%%
\section{Implementation} \label{sec:impl}

\par We present our methodology of discovering the bottle-necks of 
\texttt{Snap v6.0} implementation, and our approach to alleviate the performance
impact via parallelism. Furthermore, to avoid having to constantly spawn and
destroy threads, we designed a generic thread pool to coordinate and 
synchronize these threads to avoid the overhead introduced by thread creation
and deallocation.


\subsection{Performance Analysis of \textbf{SNAP}}

% Snap (version 6.0) is a large code base with about \texttt{111k} 
% lines of \texttt{C++} code.

\par We use \texttt{perf}\citep{linux-perf}, a performance analysis tools to 
guide our rewrite process. It utilize \textbf{performance counters} for Linux,
which are a kernel-based subsystem that provide a framework for all things 
performance analysis. This subsystem sends regular timer interrupts to the
running process, and back trace of the process's call-stack for the instruction
being executed in its interrupt handler. Such profiling trace provides us with 
runtime statistics of which instructions and functions consume more processor 
time, and ultimately insights
into which part of code to optimize for highest performance improvement.

\begin{table}
    \centering
    \caption{The profiler output for the \texttt{agmfitmain} program.}
    \label{tab:agmfitbase}
    \input{images/agmfit-base} % Generated by Makefile.
\end{table}

\paragraph{Example} A runtime trace of the \texttt{agmfitmain} program
\footnote{An executable for overlapping community detection task, see 
\url{https://github.com/Fudanyrd/snap/blob/master/examples/agmfit/ReadMe.txt}.}
is shown in Table~\ref{tab:agmfitbase}. We observe that 
\texttt{TAGMFit::SelectLambdaSum}\footnote{Link \url{https://github.com/Fudanyrd/snap/blob/master/snap-adv/agmfit.cpp\#L1007}.}
consumes a considerable proportion of CPU time.
Its code logic is iterating over a hash table and computing the 
$\sum_{k\in C_{uv}} \log (1-p_{k})$ discussed in \citep{yang2012structure}.
Unfortunately, this method is almost impossible to parallelize; due to the
fact that hash table is a sparse data structure, dividing this method into 
several tasks cannot be done efficiently in $O(1)$ time
\footnote{Iterating an array or vector can be divided because one can use integer-index.}.

\subsection{Thread Pool Implementation}

\par Indeed, the \snap library has some parallelized code, mainly 
with the \texttt{openmp} framework\citep{openmp,embedded-hps}. An example
of such optimization is shown in Figure~\ref{code:openmp}. As part of the page-rank
implementation, this snippet copies results from \texttt{PRankV} to \texttt{PRankH}.
Instead of using one core to move the data, it uses \texttt{openmp} parallelism; 
each  thread copies a disjoint part of data
\footnote{Permalink 
\url{https://github.com/Fudanyrd/snap/blob/b62ece5fe169686a81dbc1a58669722d81114b33/snap-core/centr.cpp\#L518}}.

\begin{figure}[ht]
    \centering
\begin{lstlisting}[frame=tlbr]
    /* Code snippet, line 515, snap-core/centr.cpp */
    #pragma (*\namedplaceholder{omp parallel}*) for schedule(dynamic,100000)
    for (int i = 0; i < NNodes; i++) {
        TNEANet::TNodeI NI = NV[i];
        PRankH[i] = PRankV[NI.GetId()];
    }

    return 0;
\end{lstlisting}
    \caption{An example of \texttt{openmp}-introduced parallelism.}
    \label{code:openmp}
\end{figure}

\par Unfortunately, if the above code snippet is executed intensely, the 
performance improvements introduced by \texttt{openmp} will diminish. Since 
the profiler cannot provide us with the reason for this phenomenon, we can
only assume from empirical evidence that it is because of the thread creation 
and deallocation overhead. An illustrative code snippet and its runtime
statistics is shown in Figure~\ref{code:ompdemo} and Figure~\ref{fig:ompdemo} respectively,
to support this claim\footnote{Compiled and linked with \texttt{-O1 -fopenmp -lrt}}.
Therefore, we feel that keeping all these thread in a global pool is necessary;
it avoids such runtime overhead by creating threads at program initialization,
and destroys them before exiting.

\begin{figure}[ht]
    \centering
\begin{lstlisting}[frame=tlbr]
#include <omp.h>

int main(int argc, char **argv) {
  int fd = open("/dev/null", O_WRONLY);
  /* Here N is provided by command-line arguments. 
   * It controls how many times the sum operation
   * are executed.
   */
  for (int run = 0; run < (*\namedplaceholder{N}*) ; run++) {
    int total = 0;
    #pragma (*\namedplaceholder{omp parallel}*) for
    for (int i = 0; i < 1024; i++) {
      total += rand();
    }
    /* Avoid compiler optimizing out the total,
     * and introduce lower overhead.
     */
    (void) write(fd, &total, sizeof(total));
  }

  close(fd);
  return 0;
}
\end{lstlisting}
    \caption{Demo program to illustrate overhead of threading.}
    \label{code:ompdemo}
\end{figure}

\begin{figure}[ht]
    \centering
    \includegraphics[width=0.5\textwidth]{images/ompdemo.pdf}
    \caption{The Execution Time of Program~\ref{code:ompdemo}. \\
        The \texttt{Ideal} line illustrates the execution time when the threading
        approach has no overhead. \\
        We conclude that the overhead of threading increases with scaling of $N$.
        When $N = 10^5$, the overhead reaches \textbf{58.2\%}. }
    \label{fig:ompdemo}
\end{figure}

\par The idea of the thread pool is to reserve several worker threads; when the
main thread needs to perform some intense computations, the pool will wake up
these workers to perform their assigned tasks separately. The pseudo code of how
we want to parallelize the snippet~\ref{code:ompdemo} with a thread pool
is shown in Figure~\ref{code:tpdemo}. Task assignment is done by the \texttt{addTask}
method; retrieving result is achieved by \texttt{waitFor}.

\begin{figure}[ht]
    \centering
\begin{lstlisting}[frame=tlbr]
/* Global thread pool. */
extern (*\namedplaceholder{ThreadPool}*) tpool;

/* Returns sum of `n` random numbers. */
static int worker(int n) {
    int ret = 0;
    for (int i = 0; i < n; i++) ret += rand();
    return ret;
}

int main() {
    /* Divide the task evenly into K parts; 
     * add tasks the pool for worker thread to perform. 
     */
    for (int t = 0; t < K; t++) (*\namedplaceholder{tpool.addTask}*)(worker, N / K);

    int total = 0, claimed = 0;
    task_t *finished;
    /* Aggregate the result of each task. */
    while (finished = (*\namedplaceholder{tpool.waitFor}*)()) 
        ((total += TASK_RESULT(finished)), (claimed += 1));
    
    /* Should claim K results. */
    assert(claimed == K);
}
\end{lstlisting}
\end{figure}

%%%%%%%%%%%%%%%%%%%%%%%%%%%%%%%%%%%%%%%%%%%%%%%%%%%%%%%%%%%%%%%%%%%%%%%%%%%%%%%%

