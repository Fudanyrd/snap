%%%%%%%%%%%%%%%%%%%%%%%%%%%%%%%%%%%%%%%%%%%%%%%%%%%%%%%%%%%%%%%%%%%%%%%%%%%%%%%%
\section{Evaluation}\label{sec:evaluation}

\paragraph{Experiment Setup}. We ran our experiments on a physical
\texttt{Ubuntu-22.04} server with 8 Intel i5 cores and 16 GB RAM. We use 
\texttt{x86\_64-linux-gnu-g++} version \texttt{11.4.0} as our \texttt{C++} compiler.
We build the \textbf{SNAP} project in debug mode
\footnote{This means that assertions and debug outputs in the code are enabled.}
and with highest possible compiler optimization  level(\texttt{-O3}), and 
enabled link-time optimization(\texttt{-flto}).
For our parallel executions, we use $6$ worker threads when creating the thread
pool
\footnote{Along with the main(coordinator) thread, a total of 7 threads 
are used.}. For all of our experiments, we ensure that there is a gap of at
least 10 minutes between each execution to mimic the effect of rebooting
\footnote{Ideally, we should reboot before each execution. Since we use a
physical server, this posits much manual efforts.}.

\paragraph{Evaluation Metrics}. We expect our implementation to achieve 
performance gains via parallelized execution. Therefore, we compare the 
\textbf{execution time} and \textbf{CPU utility rate} of our 
manually-optimized code. Ideally, a thread pool with $N$ workers should yield
a CPU utility rate of $N\times 100\%$, and a speed-up of $N$\texttt{x}. However,
we did not try to evaluate the \texttt{scalability}, a desirable property of 
parallel algorithms, because of the limitation of computation capabilities of
 our physical server.

\subsection{Community-Affiliation Graph Model}

\par Community-Affiliation Graph Model\citep{yang2012structure,yang2012community} 
is a conceptual model of community network, capturing the overlapping nature of 
network communities.

\paragraph{Performance on Complete Graph} We measured the execution time of 
both original version of the \texttt{agmfitmain} program
\footnote{A frontend of AGM algorithm, see \url{https://github.com/Fudanyrd/snap/blob/master/examples/agmfit/agmfitmain.cpp}.}
 and our optimized version. For its input data, we constructed an undirected,
complete graph with $200$ vertices and $19900$ edges. This medium-size graph 
is large enough  for performance-benchmarking purposes.
% to demonstrate the difference between two implementations.
We ran each compiled program $8$ times.
The result is shown in Figure~\ref{fig:agmcomplete}. We also measured the CPU utility 
rate of both our optimized program  against original ones, the result is 
shown in Figure~\ref{fig:agmcpu}. We observe that our implementation make
good use of computation resources available, and completes the 
community detection task faster.

\begin{figure*}[ht]
    \centering
    \begin{subfigure}[b]{0.45\textwidth}
        \includegraphics[width=\textwidth]{images/agmcomplete.pdf}
        \caption{Execution Time Comparison of \texttt{agmfitmain}. }
        \label{fig:agmcomplete}
    \end{subfigure}
    \begin{subfigure}[b]{0.45\textwidth}
        \includegraphics[width=\textwidth]{images/agmcpu.pdf}
        \caption{CPU Status of Original and Our Implementation.}
        \label{fig:agmcpu}
    \end{subfigure}
    \caption{Performance Evaluation}
\end{figure*}

\paragraph{Program Output Visualization} We selected three datasets:
the \texttt{NCAA football teams}\footnote{This is the testing dataset 
for AGM algorithm, see \url{https://github.com/Fudanyrd/snap/blob/master/examples/agmfit/ReadMe.txt}. }, 
the \texttt{Facebook EGO Network}\citep{egonetwork}, 
and the \texttt{GEM Facebook page network}\citep{gemsec} as test input
for our optimized \texttt{agmfitmain} program. We observe that our program
yields similar\footnote{As stated in the Related Work section, AGM
is essentially a convex optimization problem, hence there is no guarantee
that each execution produces identical output. }
outputs as original program, therefore increasing 
the confidence that our manual optimizations are valid. 
We use Gephi\footnote{\url{https://github.com/gephi/gephi.git}} as 
our visualization tool, and \texttt{Force Atlas} as the layout style. Moreover,
we modified the graph dump module(\texttt{TAGMUtil}) and assign a color
to nodes belonging to the same community\footnote{See \texttt{TAGMUtil::DefaultColorTable}
method}.
The visualization of the three datasets are shown in Figure~\ref{fig:agmviz}. 

\begin{figure*}[ht]
    \centering
    \begin{subfigure}[b]{0.32\textwidth}
        \includegraphics[width=\textwidth]{viz/football.pdf}
        % agmfitmain -c:3
        \caption{NCAA Football Teams}
    \end{subfigure}
    \begin{subfigure}[b]{0.32\textwidth}
        \includegraphics[width=\textwidth]{viz/facebook.pdf}
        % agmfitmain -l:facebook/0.feat -i:gemsec/0.edges -c:7
        \caption{Facebook EGO Network}
    \end{subfigure}
    \begin{subfigure}[b]{0.32\textwidth}
        \includegraphics[width=\textwidth]{viz/gem.pdf}
        % agmfitmain -l:gemsec/tvshow.feat -i:gemsec/tvshow.edges -c:3
        \caption{GEM Facebook Page Network}
    \end{subfigure}
    \caption{Visualization of Community Detection Result.}
    \label{fig:agmviz}
\end{figure*}

%%%%%%%%%%%%%%%%%%%%%%%%%%%%%%%%%%%%%%%%%%%%%%%%%%%%%%%%%%%%%%%%%%%%%%%%%%%%%%%%

